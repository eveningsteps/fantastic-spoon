\documentclass[a4paper, 12pt]{article}
\usepackage[left=2cm,right=2cm,top=2cm,bottom=2cm,bindingoffset=0cm]{geometry}
\usepackage[utf8]{inputenc}
\usepackage[english, russian]{babel}
\usepackage{amssymb, latexsym, amsmath}
\usepackage{indentfirst}
\usepackage{graphicx}
\usepackage{citehack}
\usepackage{tabularx}
\usepackage{listings}
\usepackage{pdfpages}
\usepackage{tikz}
\usepackage{pgfplots}

\lstloadlanguages{C++}
\lstset{extendedchars=false,
	breaklines=true,
	breakatwhitespace=true,
	keepspaces = true,
	tabsize=4
}

\begin{document}
\begin{titlepage}

\newpage

\begin{center}
Московский Авиационный Институт \\*
(национальный исследовательский университет) \\*

\vspace{2em}

Факультет прикладной математики и физики \\*
Кафедра вычислительной математики и программирования

\vspace{20em}

\Large \textbf{Лабораторная работа 2 \\*
по курсу <<Обработка естественых языков>>} \\*
II семестр

\end{center}

\vspace{15em}

\hspace{30em}\vbox{
	\hbox{Студент Данилычев И.\,А.}
	\hbox{Группа 8О-106М}
}

\vspace{\fill}

\begin{center}
Москва, 2016
\end{center}

\end{titlepage}

\newpage


\subsection*{Постановка задачи}
Необходимо реализовать булев поиск по обратному индексу, построенному в ЛР 1 -- 2.

На стандартный ввод программе поступают запросы, содержащие токены и ключевые слова {\tt and, or, not}. В ответ на каждый запрос программа выводит список документов, удовлетворяющих запросу, а также время выполнения запроса.


\subsection*{Алгоритм}
Для разбора запроса используется алгоритм сортировочной станции. После этого запрос, переведённый в постфиксную нотацию, выполняется.

При выполнении запроса учитывается частотность слов для оптимизации запросов вида {\tt A and B and C and ...}. Для этого на этапе разбора несколько двуместных операторов {\tt and} заменяются на один $N$-местный, что позволяет позднее отсортировать слова по частоте и тем самым ускорить пересечение списков документов за счёт более быстрого сужения результирующего списка.


\subsection*{Исходный код}
\lstinputlisting{../src/searcher.d}


\newpage
\subsection*{Результат выполнения}
\lstinputlisting{./result.txt}


\subsection*{Выводы}
Время выполнения запроса с несколькими {\tt and} или {\tt or} можно оптимизировать, отсортировав списки вхождений по количеству документов. К примеру, для запроса ''и and в and я'' ускорение составило $1 - \dfrac{0.323}{0.507} \approx 36\%$.

\end{document}

