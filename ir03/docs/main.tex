\documentclass[a4paper, 12pt]{article}
\usepackage[left=2cm,right=2cm,top=2cm,bottom=2cm,bindingoffset=0cm]{geometry}
\usepackage[utf8]{inputenc}
\usepackage[english, russian]{babel}
\usepackage{amssymb, latexsym, amsmath}
\usepackage{indentfirst}
\usepackage{graphicx}
\usepackage{citehack}
\usepackage{tabularx}
\usepackage{listings}
\usepackage{pdfpages}
\usepackage{tikz}
\usepackage{pgfplots}
\usepackage{multirow}

\lstloadlanguages{C++}
\lstset{extendedchars=false,
	breaklines=true,
	breakatwhitespace=true,
	keepspaces = true,
	tabsize=4
}

\begin{document}
\begin{titlepage}

\newpage

\begin{center}
Московский Авиационный Институт \\*
(национальный исследовательский университет) \\*

\vspace{2em}

Факультет прикладной математики и физики \\*
Кафедра вычислительной математики и программирования

\vspace{20em}

\Large \textbf{Лабораторная работа 9 \\*
по курсу <<Информационный поиск>>} \\*
II семестр

\end{center}

\vspace{15em}

\hspace{30em}\vbox{
	\hbox{Студент Данилычев И.\,А.}
	\hbox{Группа 8О-106М}
}

\vspace{\fill}

\begin{center}
Москва, 2016
\end{center}

\end{titlepage}


\newpage


\subsection*{Постановка задачи}
Использовать при индексировании документов (ЛР 1--2) внешний стеммер и сравнить параметры полученного индекса с первоначальными.


\subsection*{Исходный код}

\subsubsection*{Обёртка (C++)}
\lstinputlisting{../src/glue.cpp}

\subsubsection*{Обёртка (D)}
\lstinputlisting{../src/wrapper.d}

\newpage
\subsubsection*{Индексирование}
\lstinputlisting{../src/main.d}


\newpage
\subsection*{Результат выполнения}
\lstinputlisting{./result.txt}

\vspace{1em}

\begin{tabular}{|c|c|c|c|c|}
\hline
Произведение & Стеммер & Терминов & Средняя длина & Время работы \\
\hline
\multirow{2}{*}{<<Анна Каренина>>}   & -                 & 33634 & 8.21 & 530 мс  \\
\cline{2-5}
                                     & Oleander (Porter) & 13726 & 6.53 & 1332 мс \\
\hline
\multirow{2}{*}{<<Воскресение>>}     & -                 & 28939 & 8.98 & 344 мс  \\
\cline{2-5}
                                     & Oleander (Porter) & 14677 & 8.36 & 792 мс  \\
\hline
\multirow{2}{*}{<<Исследование...>>} & -                 & 14517 & 8.78 & 220 мс  \\
\cline{2-5}
                                     & Oleander (Porter) & 7253  & 7.63 & 460 мс  \\
\hline
\end{tabular}


\vspace{1em}

\subsection*{Выводы}
Количество терминов в текстах после стемминга сократилось примерно в два--три раза, а размер индекса -- в три--четыре раза (для текста <<Исследование догматического богословия>> --- с 2.8 мегабайт до 698 килобайт).
\end{document}

