\documentclass[a4paper, 12pt]{article}
\usepackage[left=2cm,right=2cm,top=2cm,bottom=2cm,bindingoffset=0cm]{geometry}
\usepackage[utf8]{inputenc}
\usepackage[english, russian]{babel}
\usepackage{amssymb, latexsym, amsmath}
\usepackage{indentfirst}
\usepackage{graphicx}
\usepackage{citehack}
\usepackage{tabularx}
\usepackage{listings}
\usepackage{pdfpages}
\usepackage{tikz}
\usepackage{pgfplots}
\usepackage{multirow}

\lstloadlanguages{C++}
\lstset{extendedchars=false,
	breaklines=true,
	breakatwhitespace=true,
	keepspaces = true,
	tabsize=4
}

\begin{document}
\begin{titlepage}

\newpage

\begin{center}
Московский Авиационный Институт \\*
(национальный исследовательский университет) \\*

\vspace{2em}

Факультет прикладной математики и физики \\*
Кафедра вычислительной математики и программирования

\vspace{20em}

\Large \textbf{Лабораторная работа 2 \\*
по курсу <<Обработка естественых языков>>} \\*
II семестр

\end{center}

\vspace{15em}

\hspace{30em}\vbox{
	\hbox{Студент Данилычев И.\,А.}
	\hbox{Группа 8О-106М}
}

\vspace{\fill}

\begin{center}
Москва, 2016
\end{center}

\end{titlepage}

\newpage


\subsection*{Постановка задачи}
Реализовать алгоритм пересечения с прыжками по индексу, построенному в ЛР 1--2. Подсчитать количество сравнений координат для различной длины прыжка, сравнить его с количеством сравнений при работе с обычным индексом.


\subsection*{Исходный код}
\lstinputlisting{../src/searcher.d}


\subsection*{Результат выполнения}
\lstinputlisting{./result.txt}

\vspace{1cm}

\begin{tabular}{|c|c|c|}
\hline
Запрос         & Длина прыжка & Сравнения \\
\hline
\multirow{5}{*}
{и and анна}   & -            & 4236 \\
               & 2            & 2682 \\
               & 4            & 1999 \\
               & 6            & 1852 \\
               & 8            & 1868 \\
\hline
\multirow{5}{*}
{и and он}     & -            & 5030 \\
               & 2            & 4891 \\
			   & 4            & 4800 \\
			   & 6            & 4924 \\
			   & 8            & 4946 \\
\hline
\multirow{5}{*}
{тебя and вот} & -            & 504  \\
               & 2            & 449  \\
			   & 4            & 436  \\
			   & 6            & 473  \\
			   & 8            & 479  \\
\hline
\end{tabular}

\vspace{1cm}

\subsection*{Выводы}
Наибольший прирост производительности приносят малые длины прыжков (от 2 до 6). Чем меньше пересечение списков, тем сильнее длина прыжка влияет на количество сравнений, которое, впрочем, в определённый момент вновь начинает расти.
\end{document}
