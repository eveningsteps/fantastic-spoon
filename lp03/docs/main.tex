\documentclass[a4paper, 12pt]{article}
\usepackage[left=2cm,right=2cm,top=2cm,bottom=2cm,bindingoffset=0cm]{geometry}
\usepackage[utf8]{inputenc}
\usepackage[english, russian]{babel}
\usepackage{amssymb, latexsym, amsmath}
\usepackage{indentfirst}
\usepackage{graphicx}
\usepackage{citehack}
\usepackage{tabularx}
\usepackage{listings}
\usepackage{pdfpages}
\usepackage{tikz}
\usepackage{pgfplots}

\lstloadlanguages{C++}
\lstset{extendedchars=false,
	breaklines=true,
	breakatwhitespace=true,
	keepspaces = true,
	tabsize=4
}

\begin{document}
\begin{titlepage}

\newpage

\begin{center}
Московский Авиационный Институт \\*
(национальный исследовательский университет) \\*

\vspace{2em}

Факультет прикладной математики и физики \\*
Кафедра вычислительной математики и программирования

\vspace{20em}

\Large \textbf{Лабораторная работа 2 \\*
по курсу <<Обработка естественых языков>>} \\*
II семестр

\end{center}

\vspace{15em}

\hspace{30em}\vbox{
	\hbox{Студент Данилычев И.\,А.}
	\hbox{Группа 8О-106М}
}

\vspace{\fill}

\begin{center}
Москва, 2016
\end{center}

\end{titlepage}

\newpage


\subsection*{Постановка задачи}
Получить наиболее вероятные коллокации для некоторого большого текстового корпуса.

\subsection*{Алгоритм}
В качестве корпуса был взят архив новостей <<РИА Новости>> за 2007 год (100 тысяч предложений). Вес коллокаций, представлявших собой биграммы, не содержащие стоп-слов, рассчитывался по критерию Стьюдента с уровнем значимости $0.05$.

\subsection*{Исходный код}
\lstinputlisting{../src/collocations.d}

\subsection*{Результат выполнения}
\lstinputlisting{./result.txt}

\subsection*{Выводы}
В 50 самых часто встретившихся коллокаций входят временные характеристики (<<(в) $N$ году>>), денежные суммы (тысячи и миллионы долларов). Кроме этого, есть также несколько фамилий и имён известных личностей и названий организаций, а также устойчивые обороты вроде <<речь идёт>>, <<точка зрения>> и т.п., которые и являются коллокациями. Возможно, при выборе другого критерия (напр., Пирсона) <<настоящих>> коллокаций было бы больше.
\end{document}

