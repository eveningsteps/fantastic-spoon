\documentclass[a4paper, 12pt]{article}
\usepackage[left=2cm,right=2cm,top=2cm,bottom=2cm,bindingoffset=0cm]{geometry}
\usepackage[utf8]{inputenc}
\usepackage[english, russian]{babel}
\usepackage{amssymb, latexsym, amsmath}
\usepackage{indentfirst}
\usepackage{graphicx}
\usepackage{citehack}
\usepackage{tabularx}
\usepackage{listings}
\usepackage{pdfpages}
\usepackage{tikz}
\usepackage{pgfplots}
\usepackage{verbatim}

\lstloadlanguages{C++}
\lstset{extendedchars=false,
	breaklines=true,
	breakatwhitespace=true,
	keepspaces = true,
	tabsize=4
}

\begin{document}
\begin{titlepage}

\newpage

\begin{center}
Московский Авиационный Институт \\*
(национальный исследовательский университет) \\*

\vspace{2em}

Факультет прикладной математики и физики \\*
Кафедра вычислительной математики и программирования

\vspace{20em}

\Large \textbf{Лабораторная работа 2 \\*
по курсу <<Обработка естественых языков>>} \\*
II семестр

\end{center}

\vspace{15em}

\hspace{30em}\vbox{
	\hbox{Студент Данилычев И.\,А.}
	\hbox{Группа 8О-106М}
}

\vspace{\fill}

\begin{center}
Москва, 2016
\end{center}

\end{titlepage}

\newpage


\subsection*{Постановка задачи}
Реализовать исправление опечаток.

\subsection*{Алгоритм}
Получившийся спеллчекер работает следующим образом. Вначале для каждого слова запроса генерируются все возможные кандидаты  с помощью каждой из четырёх операций:

\begin{itemize}
	\item Удаление произвольного символа;
	\item Добавление произвольного символа;
	\item Перестановка соседних символов местами;
	\item Замена произвольного символа на произвольный.
\end{itemize}

Из кандидатов отбираются имеющиеся в обучающем корпусе, после чего с помощью алгоритма Витерби и сглаживающей языковой модели (Лидстон + биграммы) строится наиболее вероятная исправленная фраза.

\subsection*{Исходный код}
\lstinputlisting{../src/spellchecker.d}
\lstinputlisting{../src/main.d}

\subsection*{Результат выполнения}
\verbatiminput{./result.txt}

\subsection*{Выводы}
Алгоритм Витерби позволяет избавиться от экспоненциальной сложности при поиске наиболее вероятного варианта исправления; время работы алгоритма составляет $O(T \cdot {|S|}^2)$, где $T$ --- число наблюдений (в данном случае --- количество слов в запросе), а $S$ --- множество состояний (слов-кандидатов).

Возможно, вместо генерации кандидатов <<втупую>> стоило воспользоваться нечётким поиском, что могло бы сократить (а могло бы и не сократить, в зависимости от числа кандидатов, не обнаруженных в тексте) время работы.

Что касается результатов, то у языковой модели есть свои проблемы, например, исправление менее популярных биграмм на более популярные (см. исправление к <<после удавшегося поукшения>>). От этого можно избавиться, как запрещая исправлять присутствующие в тексте биграммы, так и увеличивая порядок модели.

\end{document}

